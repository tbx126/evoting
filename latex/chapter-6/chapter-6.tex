\chapter{Conclusions and Future Work}

\begingroup
\justifying
\setlength{\parindent}{0pt}
\setstretch{1.5}
\setlength{\parskip}{0.5\baselineskip}
\titlespacing{\chapter}{0pt}{0pt}{0pt}
\titlespacing{\section}{0pt}{0pt}{0pt}

\section{Conclusions}

This dissertation implemented and evaluated a blockchain-based e-voting system in which privacy and verifiability are enforced by cryptographic protocol rather than by institutional trust. The core architecture combines three mechanisms: homomorphic ElGamal encryption on BabyJubJub for ballot confidentiality, Groth16 zero-knowledge proofs for vote legality and tally correctness, and commitment-based Merkle inclusion proofs for anonymous voter-side auditability.

The principal engineering result is that all three mechanisms can be integrated into a coherent, browser-to-chain workflow using mainstream tooling: Circom and snarkjs for circuit development, Hardhat and Solidity for contract deployment, and standard Web3 wallet APIs for voter interaction. Crucially, this integration does not require a trusted middleware layer---vote proof verification occurs entirely on-chain, and tally correctness is publicly attested by the submitted ZKP3 proof.

From a research perspective, the system addresses RQ1 by demonstrating that encrypted one-hot ballots and ZKP-based checks are technically feasible in a browser-to-chain pipeline without trusted middleware. It addresses RQ2 by showing that the combination of on-chain proof verification and receipt-based anonymous auditability provides verifiability that is structurally stronger than centralized log-based auditing: observers do not need to trust the operator's claim of correctness, only the cryptographic assumptions underlying the proof system. For RQ3, the gap analysis in Chapter 5 identifies five specific engineering issues that must be resolved before the prototype can be described as production-ready.

The evaluation confirmed 75 automated tests passing---32 Hardhat contract and integration tests and 43 Python cryptographic unit tests---providing a reproducible baseline for future development. Performance measurements show that on-chain costs are dominated by Groth16 verification (approximately 181,000 gas for the pairing check), and that the system is practically viable for small-to-medium elections at current Ethereum gas parameters.

Taken together, the implementation represents a functional, testable, and honest research prototype: it achieves its stated privacy and verifiability goals, documents its residual gaps explicitly, and provides a clear engineering path toward higher-assurance deployment.

\section{Completion Assessment Against Project Statement}

Against the stated project goal (``a fully functional e-voting system with verifiability and voter privacy''), the current status can be summarized as follows:
\begin{itemize}
    \item \textbf{Functional prototype objective}: achieved. The end-to-end election workflow executes with verifiable vote and tally proofs plus commitment-based inclusion audit.
    \item \textbf{Research comparison objective}: achieved. The dissertation analyzes integrity, transparency, and tamper resistance relative to centralized baselines.
    \item \textbf{High-assurance production objective}: not yet achieved. Remaining hardening tasks are explicit and technically scoped.
\end{itemize}

This assessment keeps the claim boundary precise: complete as a demonstrator and dissertation artifact, incomplete as a production election platform.

\section{Limitations}

The current system has the following limitations:
\begin{itemize}
    \item administrator key custody is a central operational trust assumption;
    \item circuit configuration is fixed for a specific candidate count in current artifacts;
    \item some bindings between on-chain public signals and off-chain payloads require reinforcement;
    \item coercion resistance and strong endpoint security are outside scope.
\end{itemize}

\section{Future Work}

The following roadmap is prioritized for completing remaining tasks:
\begin{enumerate}
    \item \textbf{Strict binding of tally results}: bind the submitted result vector to the public signals verified by Groth16 to prevent inconsistent reporting.
    \item \textbf{Encrypted payload integrity}: compute and enforce ciphertext hash binding to emitted payloads or redesign tally ingestion to avoid relying on mutable sources.
    \item \textbf{Root authenticity checks}: compare audit bundle root to on-chain \texttt{merkleRoot} in the voter verification flow.
    \item \textbf{Candidate-count enforcement}: circuits are hardcoded for two candidates; enforce this constraint at the contract level to prevent mismatched deployments from accepting proofs silently.
    \item \textbf{Governance hardening}: lifecycle-gate or remove unrestricted Merkle-root overwrite functions.
    \item \textbf{Decentralized key management}: introduce threshold encryption/decryption or multi-party custody to reduce single-admin trust.
    \item \textbf{Coercion resistance research}: explore receipt-freeness mechanisms and stronger threat models.
\end{enumerate}

Table \ref{tab:roadmap-phases} converts the future-work list into phased engineering milestones.

\begin{table}[htbp]
\centering
\small
\caption{Phased Hardening Roadmap}
\label{tab:roadmap-phases}
\begin{tabular}{L{2.2cm} L{4.6cm} L{4.8cm}}
\toprule
Phase & Priority Deliverables & Expected Outcome \\
\midrule
Near-term & Result-vector/public-signal binding, bundle-root vs on-chain-root check, event payload integrity checks & Close critical correctness gaps and strengthen trust in audit outputs \\
Mid-term & Candidate-count enforcement at contract level, lifecycle-gated Merkle root governance, deployment hardening & Reduce misuse risk and improve operational robustness \\
Long-term & Threshold key management, coercion-resistance mechanisms, formal verification extensions & Move from prototype assurance toward high-assurance production viability \\
\bottomrule
\end{tabular}
\end{table}

Overall, the project is complete as a functional prototype demonstrating verifiable and privacy-aware blockchain voting, with a clear set of engineering steps required for full high-assurance completion.

\endgroup
