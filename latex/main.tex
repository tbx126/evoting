\documentclass[12pt,a4paper]{report}

% ============================ 预加载的 LaTeX 包 ============================
\usepackage[a4paper,left=3.5cm,right=3cm,top=3cm,bottom=3cm]{geometry} % 页面边距
\usepackage[protrusion=true,expansion=true]{microtype}  % 细节优化
\usepackage{setspace}                         % 设置 1.5 倍行距
\usepackage{fancyhdr}                         % 自定义页眉页脚
\usepackage{newtxtext,newtxmath}              % 设置 Times New Roman 字体
\usepackage{amsmath, amssymb, amsthm}         % 数学支持
\usepackage{graphicx}                         % 插入图片
\usepackage{hyperref}                         % 目录超链接
\usepackage{booktabs}                         % 美化表格
\usepackage{array}                            % 处理表格对齐
\usepackage{multicol}                         % 多列布局
\usepackage[toc,page]{appendix}               % 处理附录
\usepackage[nottoc]{tocbibind}  % 让目录中包含“参考文献”和“附录”
\usepackage{caption}                          % 处理表格和图片标题格式
\usepackage{xcolor}                           % 颜色支持
\usepackage{titlesec}                         % 控制标题字体大小
\usepackage{eso-pic}                          % 添加水印
\usepackage{tikz}                             % 让水印精确定位
\usepackage{datetime}                         % 格式化日期
\usepackage{tocloft}                          % 目录格式控制
\usepackage{titlesec}                         % 标题格式控制
\usepackage{titletoc}                         % 目录格式增强
\usepackage{cite}                             % bib引用
\usepackage{lipsum}                           % 内容模拟
% \usepackage{showframe}
\usepackage{etoolbox}
\usepackage{ragged2e}

% ================================= 设置公式 ================================
\AtBeginEnvironment{equation}{\setstretch{1}}             % 设置公式1倍行距
\AtEndEnvironment{equation}{}                             % 公式结束恢复原样式
\numberwithin{equation}{section}                          % 让公式编号跟随章节
\renewcommand{\theequation}{\arabic{section}.\arabic{equation}}% 仅显示1.1

% ================================= 设置页脚 ================================
\pagestyle{fancy}                                         % 启用 fancyhdr
\fancyhf{}                                                % 清除默认的页眉和页脚
\fancyfoot[R]{\fontsize{12pt}{14pt}\selectfont \thepage}  % 右侧页脚页码 12pt
\renewcommand{\headrulewidth}{0pt}                        % 取消页眉下划线
\renewcommand{\footrulewidth}{0pt}                        % 取消页脚下划线
\fancypagestyle{plain}{                                   % 适配plain风格页面
    \fancyhf{} 
    \fancyfoot[R]{\fontsize{12pt}{14pt}\selectfont \thepage}
    \renewcommand{\headrulewidth}{0pt}
    \renewcommand{\footrulewidth}{0pt}
}

% ================================= 设置目录 ================================
% 修改目录标题格式(居中 + 18pt + 加粗 + 下划线)
\renewcommand{\contentsname}{}
\newcommand{\customtoc}{
    \clearpage
    \begin{center}
        \underline{\textbf{\fontsize{18pt}{18pt}\selectfont Table of Contents}}
    \end{center}
    \vspace{-1.2\baselineskip}
}
% 自定义 "Pages" 列标题,使其右对齐
\renewcommand{\cftaftertoctitle}{%
    \vspace{-1.5\baselineskip} % 减少 Table of Contents 后的额外空行
    \hfill\underline{Pages} % 右对齐
    \par\vskip-2\baselineskip % 调整间距,防止错位
}
%目录格式调整
% 一级标题 (chapter) 设置
\renewcommand{\cftchapfont}{\bfseries}                  % 章节标题加粗
\renewcommand\thechapter{Chapter\enspace\arabic{chapter}}       % 添加 "Chapter " 前缀
\renewcommand\thesection{\arabic{chapter}.\arabic{section}} % 解决上行代码的乱码问题
\renewcommand{\cftchapaftersnumb}{}                     % 章节编号后不添加任何内容
\setlength{\cftchapnumwidth}{5.3em}                       % 设置编号宽度
% 二级标题(section)设置
\renewcommand{\cftsecfont}{}                            % 节标题字体
\renewcommand{\cftsecpresnum}{}                         % 节编号前缀
\renewcommand{\cftsecaftersnumb}{}                      % 节编号后缀
\setlength{\cftsecindent}{0em}                          % 节的缩进
\setlength{\cftsecnumwidth}{2em}                        % 节编号宽度
\renewcommand{\cftsecdotsep}{\cftnodots}                % 移除点线
\renewcommand{\cftsecpagefont}{\color{white}}        % 移除页码
% 间距设置
\setlength{\cftbeforechapskip}{0.5em}                   % 章节前的间距
\setlength{\cftbeforesecskip}{0.5em}                    % 节前的间距

% ========================== 统一设置 \chapter ==========================
% 设置有编号的 \chapter 显示 "Chapter 1" + 换行后的标题左对齐
\titleformat{\chapter}[display]            % "Chapter 1" 独立一行,标题换行后左对齐
  {\normalfont\Huge\bfseries}  % 不倾斜,字号 Huge,加粗
  {\thechapter}                            % 显示 "Chapter 1"
  {0em}                                    % 章节编号和标题之间的间距
  {\raggedright}                           % 标题居左
% 设置无编号的 \chapter*{}(如摘要、附录等)格式,不带 "Chapter"
\titleformat{name=\chapter,numberless}[display]
  {\normalfont\Huge\bfseries}  % 不倾斜,字号 Huge,加粗
  {}                                       % 无编号时不显示 "Chapter 1"
  {-2.5em}                                 % 章节编号和标题之间的间距
  {\centering}                             % 标题居中
% 调整章节标题的间距
\titlespacing{\chapter}{0pt}{-2em}{0em}  % 影响有编号的章节
\titlespacing{name=\chapter,numberless}{0pt}{0em}{\baselineskip}  % 影响无编号的章节

% ========================== 统一设置 \section ==========================
% 设置有编号的 \section{}
\titleformat{\section}                        % 统一设置 section
  {\normalfont\fontsize{16pt}{16pt}\bfseries} % 12pt 字号行距,加粗
  {\thesection\enspace}                       % 章节编号(这里为空)
  {0pt}                                       % 章节前间距
  {\raggedright}                              % 章节后格式(这里保持默认)
% 设置无编号的 \section*{}
\titleformat{name=\section,numberless}[display]
  {\clearpage\normalfont\fontsize{16pt}{16pt}\bfseries}
  {}
  {-1em}
  {\centering}
\titlespacing{\section}{0pt}{0.5\baselineskip}{0.5\baselineskip}
\titlespacing{name=\section,numberless}{0pt}{0em}{1.5\baselineskip}

\global\sloppy  % 允许换行,减少溢出

% ======================= 统一设置 \subsection ========================
% 设置有编号的 \subsection{}
\titleformat{\subsection}    
  {\normalfont\fontsize{14pt}{14pt}\bfseries}  % 14pt 字号行距,加粗
  {\thesubsection\enspace}                     % 章节编号
  {0pt}                                        % 章节编号与标题间距
  {\raggedright}                               % 标题左对齐

% 调整 subsection 的间距
\titlespacing{\subsection}{0pt}{0.5\baselineskip}{0.5\baselineskip}


\begin{document}

\onehalfspacing % 1.5 倍行距

% ================================ 自定义颜色 ================================
\definecolor{DeepGreen}{rgb}{0.0, 0.690, 0.314}

% ================================ 自定义设置 ================================
% 左缩进为0
\setlength{\parindent}{0pt}
% 自定义日期格式
\newdateformat{mydate}{\THEDAY\ \monthname[\THEMONTH]\ \THEYEAR}
% 让图编号显示为 "Figure X.Y"
\renewcommand{\thefigure}{Figure \arabic{chapter}.\arabic{figure}}
\setlength{\cftfignumwidth}{4.5em}
\renewcommand{\figurename}{}
% 调整 listoffigures (图目录) 的格式
\makeatletter
\renewcommand{\listoffigures}{%
    \begingroup
    \vspace{-2em} % 减少上方间距
    \chapter*{\listfigurename}%
    \@starttoc{lof}%
    \endgroup
}
\makeatother
% 让表格编号显示为 "Table X.Y"
\renewcommand{\thetable}{Table \arabic{chapter}.\arabic{table}}
\setlength{\cfttabnumwidth}{4.5em}
\renewcommand{\tablename}{}
% 调整 listoftables (表目录) 的格式
\makeatletter
\renewcommand{\listoftables}{%
    \begingroup
    \vspace{-2em} % 减少上方间距
    \chapter*{\listtablename}%
    \@starttoc{lot}%
    \endgroup
}
\makeatother


% ======================== 论文前置部分(Front Matter) =======================
\pagenumbering{gobble}
\begin{titlepage}
    \centering

    \vspace*{1cm}
    \includegraphics[width=0.8\textwidth]{assets/ntu-logo.png}

    \vspace*{5.5cm}

    \textbf{\LARGE{Blockchain-Based E-Voting System with Verifiability and Voter Privacy}}

    \vfill

    \textbf{Author Name} \\
    \textbf{SCHOOL OF ELECTRICAL AND ELECTRONIC ENGINEERING} \\
    \textbf{2026}

    \vspace*{1cm}

    \newpage

    \vspace*{5.5cm}

    \textbf{\LARGE{Blockchain-Based E-Voting System with Verifiability and Voter Privacy}}

    \vspace*{4.5cm}

    \textbf{Author Name}

    \vspace*{1.5cm}

    \textbf{SCHOOL OF ELECTRICAL AND ELECTRONIC ENGINEERING}

    \vspace*{\baselineskip}

    \textbf{A DISSERTATION SUBMITTED IN PARTIAL FULFILMENT OF THE REQUIREMENTS FOR THE DEGREE OF} \\
    \textbf{MASTER OF SCIENCE IN [PROGRAMME NAME]}

    \vfill

    \textbf{2026}

    \vspace*{1cm}
\end{titlepage}
             % 论文封面
\section*{Statement of Originality}

I hereby certify that the work embodied in this dissertation is the result of original research, is free of plagiarised materials, and has not been submitted for a higher degree to any other university or institution.

\vspace*{1cm}

\noindent
\begin{center}
    \begin{tabular}{>{\centering\arraybackslash}p{6cm} p{1cm} >{\centering\arraybackslash}p{6cm}}
        \mydate\today &  &
        \begin{tikzpicture}
            \node[opacity=0.9, anchor=center] at (0,0) {\includegraphics[width=5cm]{assets/ntu-watermark.png}};
            \node[anchor=north] at (0,2) {\includegraphics[width=4cm]{signature/personal-signature.png}};
        \end{tikzpicture} \\
        \dotfill &  & \dotfill \\[0.2cm]
        Date &  & Author Name \\
    \end{tabular}
\end{center}
            % 原创性声明
\section*{Supervisor Declaration Statement}
I have reviewed the content and presentation style of this thesis and declare it is free of plagiarism and of sufficient grammatical clarity to be examined. To the best of my knowledge, the research and writing are those of the candidate except as acknowledged in the Author Attribution Statement. I confirm that the investigations were conducted in accord with the ethics policies and integrity standards of Nanyang Technological University and that the research data are presented honestly and without prejudice.

\vspace*{1cm}

\noindent
\begin{center} % 让整个表格居中
    \begin{tabular}{>{\centering\arraybackslash}p{6cm} p{1cm} >{\centering\arraybackslash}p{6cm}} % 居中对齐
        \mydate\today &  & 
        \begin{tikzpicture} % 签名&水印
            \node[opacity=0.9, anchor=center] at (0,0) {\includegraphics[width=5cm]{assets/ntu-watermark.png}};
            % 在水印上方插入手写签名图片
            \node[anchor=north] at (0,2) {\includegraphics[width=4cm]{signature/personal-signature.png}};
        \end{tikzpicture} \\ % 第一行:日期和签名水印
        \dotfill &  & \dotfill \\[0.2cm] % 第二行:虚线
        Date &  & \textcolor{red}{Supervisor Name} \\ % 第三行:日期说明 & 姓名
    \end{tabular}
\end{center} % 导师声明
\section*{Authorship Attribution Statement}

This dissertation does not contain material from papers published in peer-reviewed journals or from papers accepted at conferences in which the author is listed as a co-author.

\vspace*{1cm}

\noindent
\begin{center}
    \begin{tabular}{>{\centering\arraybackslash}p{6cm} p{1cm} >{\centering\arraybackslash}p{6cm}}
        \mydate\today &  &
        \begin{tikzpicture}
            \node[opacity=0.9, anchor=center] at (0,0) {\includegraphics[width=5cm]{assets/ntu-watermark.png}};
            \node[anchor=north] at (0,2) {\includegraphics[width=4cm]{signature/personal-signature.png}};
        \end{tikzpicture} \\
        \dotfill &  & \dotfill \\[0.2cm]
        Date &  & Author Name \\
    \end{tabular}
\end{center}
             % 著作声明

\clearpage
\centering\customtoc
% \addtocontents{toc}{\protect\setcounter{tocdepth}{-1}} % 暂时隐藏目录 (This line causes empty TOC)
\tableofcontents % 生成目录
% \addtocontents{toc}{\protect\setcounter{tocdepth}{1}}  % 恢复目录
\clearpage

\clearpage
\pagenumbering{roman}                         % 正文前使用罗马数字编号
\setcounter{page}{1}                          % 从 i 开始
\pagestyle{fancy}                             % 确保页码继续显示在右侧页脚
\chapter*{Abstract}
\addcontentsline{toc}{chapter}{Abstract}
\begingroup
\justifying
\setlength{\parindent}{0pt}
\setstretch{2} % 设置为2的时候为word1.5倍行距
\setlength{\parskip}{0.5\baselineskip}

Multihop cellular networks (MCNs) incorporate wireless ad hoc networking into traditional single-hop cellular networks (SCNs) and thus they enjoy the flexibility of ad hoc networks, while preserving the benefit of using infrastructure of SCNs. In this Thesis, we study the resource allocation problems in MCNs.

Multihop cellular networks (MCNs) incorporate wireless ad hoc networking into traditional single-hop cellular networks (SCNs) and thus they enjoy the flexibility of ad hoc networks, while preserving the benefit of using infrastructure of SCNs. In this Thesis, we study the resource allocation problems in MCNs.

\endgroup               % 摘要
\chapter*{Acknowledgement}
\addcontentsline{toc}{chapter}{Acknowledgement}
\begingroup
\justifying
\setlength{\parindent}{0pt}
\setstretch{1.5}
\setlength{\parskip}{0.5\baselineskip}

The author thanks the open-source cryptography and blockchain communities for the tools used in this work, including Hardhat, Circom, snarkjs, and related libraries. In addition, appreciation is directed towards the project supervisors and peers for the valuable feedback they provided pertaining to the protocol design, the methodology applied for testing, in addition to the execution of the security review.

\begin{flushright}
Author Name \\
\mydate\today
\end{flushright}

\endgroup
        % 致谢(可选)
\chapter*{Acronyms}
\addcontentsline{toc}{chapter}{Acronyms}

\noindent
\begin{tabular}{ll}
EVM & Ethereum Virtual Machine \\
HE & Homomorphic Encryption \\
ZKP & Zero-Knowledge Proof \\
zk-SNARK & Zero-Knowledge Succinct Non-Interactive Argument of Knowledge \\
NIZK & Non-Interactive Zero-Knowledge \\
PK & Public Key \\
SK & Secret Key \\
DApp & Decentralized Application \\
ABI & Application Binary Interface \\
DoS & Denial of Service \\
\end{tabular}
               % 术语缩写表(可选)
\chapter*{Symbols (optional)}
\addcontentsline{toc}{chapter}{Symbols (optional)}

\noindent
\begin{tabular}{ll} % 两列对齐
B & channel bandwidth in Hz \\
C & channel capacity in bps; \\
  & number of collisions in time slot $t$ \\
d & distance \\
D & minimum reuse distance \\
$D_a$ & average message access delay \\
$D_{id}$ & inter-datagram-arrival time \\
$D_{max}$ & maximum tolerable delay for voice packets \\
\end{tabular}                % 符号表(可选)
\listoffigures                                % 生成图目录
\listoftables                                 % 生成表目录

% ========================= 论文正文部分(Main Content) ========================
\clearpage
\pagenumbering{arabic}       % 正文部分改为阿拉伯数字
\setcounter{page}{1}         % 从 1 开始
\pagestyle{fancy}            % 确保页码继续显示在右侧页脚
\chapter{Introduction}
\begingroup
\justifying
\setlength{\parindent}{0pt}
\setstretch{2}
\setlength{\parskip}{0.5\baselineskip}
\titlespacing{\chapter}{0pt}{0pt}{0pt}
\titlespacing{\section}{0pt}{0pt}{0pt}

Electronic voting has long been positioned as a modernization path for democratic processes, especially in scenarios requiring rapid tallying, remote participation, and auditable outcomes. Yet, real deployment remains difficult because election systems must satisfy several properties that are usually in tension: confidentiality of individual ballots, integrity of election state transitions, universal verifiability of counting logic, and practical usability for both administrators and voters.

Traditional centralized e-voting platforms often satisfy usability and operational speed, but they rely on a trusted operator to maintain backend logs, enforce state transitions, and publish correct tally outputs. In this trust model, independent verification by voters and third-party auditors is limited by what the operator chooses to expose. Once the system is disputed, root-cause analysis frequently depends on privileged data access rather than protocol-level guarantees.

This dissertation studies a concrete alternative: a blockchain-based e-voting architecture that combines on-chain state-machine enforcement with privacy-preserving cryptography. The project integrates homomorphic ElGamal encryption, Groth16 zero-knowledge proof verification, and commitment-based Merkle inclusion proofs into one end-to-end pipeline. The goal is not merely theoretical correctness, but an implementation that can be executed, tested, and reviewed as a working engineering artifact.

\section{Problem Statement}

The central problem is how to construct a voting protocol that supports all of the following in a deployable system:
\begin{itemize}
    \item ballot confidentiality during submission, storage, and tally;
    \item correctness checks for vote formation and tally claims without revealing vote contents;
    \item tamper-evident election lifecycle management;
    \item voter-side verification that does not require disclosing identity or wallet address.
\end{itemize}

In formal terms, let $V$ denote the number of cast ballots and $N$ denote the number of candidates. The protocol must ensure that each accepted ballot encodes exactly one valid candidate choice, aggregate ciphertexts can be tallied consistently, and the published result vector is accompanied by cryptographic evidence sufficient for independent validation.

\section{Motivation}

Three observations motivate this work.

First, blockchain ledgers provide append-only event logs and deterministic contract execution semantics, reducing the attack surface for silent state rewrites \cite{nakamoto2008,wood2014}. Second, modern zk-SNARK systems provide practical proof sizes and on-chain verification costs compatible with smart-contract workflows \cite{groth2016}. Third, circuit-friendly hash functions and Merkle structures enable lightweight post-election inclusion checks for large ballot sets \cite{poseidon2019,merkle1988}.

The combination of these properties suggests a practical design space where privacy and verifiability can be treated as co-equal requirements rather than trade-offs resolved by institutional trust.

\section{Research Objectives}

This dissertation has five concrete objectives:
\begin{enumerate}
    \item design a contract-level election state machine with explicit lifecycle constraints;
    \item implement private ballot casting using one-hot ElGamal encryption on BabyJubJub;
    \item enforce vote legality and tally correctness through Groth16 proof verification;
    \item implement an anonymous voter audit path through commitment receipts and Merkle proofs;
    \item evaluate requirement completion against the project brief and identify remaining hardening gaps.
\end{enumerate}

\section{Research Questions}

The implementation and evaluation are organized around the following questions:
\begin{itemize}
    \item \textbf{RQ1}: Can encrypted one-hot ballots and ZKP-based checks be integrated into a browser-to-chain workflow without trusted middleware?
    \item \textbf{RQ2}: Does the resulting pipeline provide verifiability that is stronger than centralized log-based auditing?
    \item \textbf{RQ3}: Which security guarantees are already achieved in code, and which remain partially satisfied due to engineering gaps?
\end{itemize}

Table \ref{tab:rq-mapping} maps each research question to the main validation locus in this dissertation.

\begin{table}[htbp]
\centering
\small
\caption{Research Question to Validation Mapping}
\label{tab:rq-mapping}
\begin{tabular}{L{2.0cm} L{5.0cm} L{5.8cm}}
\toprule
RQ & Evaluation Focus & Primary Evidence Location \\
\midrule
RQ1 & End-to-end integration feasibility of client crypto, proof generation, and contract verification & Chapter 4 implementation workflow and Chapter 5 scenario validation \\
RQ2 & Comparative verifiability gain over centralized e-voting & Chapter 2 positioning and Chapter 5 security analysis \\
RQ3 & Gap analysis between functional completion and high-assurance completion & Chapter 5 outstanding gaps and Appendix B checklist \\
\bottomrule
\end{tabular}
\end{table}

\section{Scope and Boundaries}

The implemented system targets a single-election deployment model with an administrator-controlled tally key. The solution is designed as a high-confidence prototype and research demonstrator, not a production-national-election platform.

Out-of-scope items include coercion resistance, threshold decryption ceremonies, decentralized governance for election administration, and formal verification of contract bytecode. These are addressed as future work rather than claimed as completed features.

\section{Contributions}

This dissertation contributes the following:
\begin{itemize}
    \item an executable cross-stack architecture connecting Solidity contracts, Circom circuits, JavaScript frontend proving, and Python cryptographic validation;
    \item a vote-submission mechanism that binds commitment and ciphertext hash into public proof signals;
    \item a tally pipeline that integrates homomorphic aggregation, ZKP3 verification, and Merkle-root publication;
    \item an anonymous verification procedure in which the voter proves ballot inclusion using only receipt material and an exported audit bundle;
    \item a requirement-driven completion assessment with explicit unresolved risks.
\end{itemize}

\section{Dissertation Structure}

Chapter 2 reviews prior work and cryptographic foundations. Chapter 3 defines the system model, protocol flow, and security assumptions. Chapter 4 details implementation decisions and module-level behavior. Chapter 5 reports evaluation outcomes, performance measurements, requirement traceability, and risk analysis. Chapter 6 concludes and proposes a prioritized hardening roadmap.

\section{Chapter Summary}

This chapter established the dissertation problem context, defined explicit research questions, and set realistic scope boundaries for interpreting security claims. It also introduced contribution statements that are later validated against implementation and evaluation evidence.

\endgroup
  % 第一章
\chapter{Literature Review}
\begingroup
\justifying
\setlength{\parindent}{0pt}
\setstretch{1.5}
\setlength{\parskip}{0.5\baselineskip}
\titlespacing{\chapter}{0pt}{0pt}{0pt}
\titlespacing{\section}{0pt}{0pt}{0pt}

\section{Security Requirements of E-Voting}

A credible e-voting system must satisfy at least four properties: eligibility control, ballot secrecy, integrity of counting, and auditability. Classical protocol literature adds finer distinctions---individual verifiability (a voter confirms their own ballot was counted), universal verifiability (any observer can check tally consistency), and receipt-freeness or coercion resistance \cite{benaloh1994,chaum2004,juels2005}.

In practice, these requirements split across cryptographic and operational controls. Cryptography can confirm that certain relations hold. It cannot, however, guarantee endpoint security, key governance, or anti-coercion behaviour in an uncontrolled environment. That gap matters when reading claims of ``secure voting''.

\section{Centralized E-Voting Systems}

Conventional centralized architectures handle the election lifecycle through web services and database transactions. Mature tooling and low operational complexity give these systems a practical edge; incident response also stays inside one trust boundary. The limitations are harder to overlook:
\begin{itemize}
    \item privileged insiders can alter records or logs if access controls slip;
    \item replaying state transitions independently requires a full data export, which is not always available;
    \item dispute resolution falls back on institutional trust rather than any verifiable record.
\end{itemize}

These gaps motivate replacing trust-in-operator with trust-in-protocol.

\section{Blockchain-Based Voting Approaches}

Blockchain infrastructures maintain immutable append-only logs with deterministic execution rules \cite{nakamoto2008,wood2014}. Election state transitions and event traces are transparently recorded. Smart contracts enforce lifecycle invariants directly---there is no hidden backend to trust.

Transparency cuts both ways, though. If vote semantics appear in plaintext, confidentiality is permanently lost. A privacy-preserving blockchain voting system must therefore pair ledger transparency for control flow with cryptographic concealment for ballot content.

\section{Homomorphic Encryption for Private Tally}

ElGamal encryption on discrete-log groups supports additive homomorphism over encoded messages \cite{elgamal1985}. Each ballot is encoded as a one-hot vector; component-wise aggregation of ciphertexts gives encrypted candidate totals. Decrypting only the aggregates avoids exposing individual ballots.

Two practical advantages follow. Tallying cost scales linearly with vote count using simple point operations. Privacy holds as long as secret-key exposure and plaintext side channels stay controlled---a weaker guarantee than threshold decryption, but achievable without ceremony infrastructure.

\section{Zero-Knowledge Proofs in Voting}

A zero-knowledge proof (ZKP) is a cryptographic protocol in which a prover convinces a verifier that a statement is true without disclosing any information beyond the statement's truth itself \cite{goldwasser1989}. In e-voting, this capability is foundational: a voter must prove that their ballot is correctly formed---encoding exactly one valid candidate choice---without revealing which candidate was selected; an administrator must prove that a tally is correctly decrypted without revealing the secret key. ZKPs therefore allow public verifiability and ballot privacy to coexist.

This dissertation uses Groth16, a zk-SNARK (Succinct Non-interactive ARgument of Knowledge), because its on-chain verification cost is fixed regardless of circuit complexity and proof size is constant at three elliptic-curve points \cite{groth2016}. Circom and snarkjs provide the implementation pipeline from circuit specification to deployable Solidity verifier contracts \cite{circomdocs,snarkjs}.

\subsection{The Problem ZKPs Address}

Classical proof systems require the witness (secret data) to be transmitted to the verifier. In e-voting this is unacceptable: revealing a candidate index to prove ballot validity destroys ballot secrecy; revealing a decryption key to prove tally correctness collapses key governance.

ZKPs separate the \emph{claim} from the \emph{evidence}. Let $x$ be a public statement and $w$ be a private witness. Define a relation $\mathcal{R}$ such that $(x, w) \in \mathcal{R}$ means ``$w$ is a valid witness for $x$''. A ZKP system allows a prover who knows $w$ to produce a short artifact $\pi$ that convinces any verifier that $(x, w) \in \mathcal{R}$, without $\pi$ leaking any information about $w$.

In this project, two instances of this pattern apply:
\begin{itemize}
    \item \textbf{Vote proof}: the public statement and witness are
    \begin{gather*}
    x = (\mathit{commitment},\; H_{ct},\; pk_X,\; pk_Y) \\
    w = (\mathit{candidateId},\; \mathit{salt},\; r_0, \dots, r_{N-1})
    \end{gather*}
    The relation $\mathcal{R}$ requires that the commitment is correctly formed from the candidate index and salt, and that the ciphertexts encode the corresponding one-hot vector under the public key.
    \item \textbf{Tally proof}: $w = sk$, with public statement
    \begin{equation*}
        \begin{split}
        x = (&pk_X,\; pk_Y,\; \bar{C1}^{X}_0,\; \bar{C1}^{Y}_0,\; \dots,\; \bar{C2}^{X}_{N-1},\; \bar{C2}^{Y}_{N-1},\\
             &R^{X}_0,\; R^{Y}_0,\; \dots,\; R^{X}_{N-1},\; R^{Y}_{N-1},\; V)
        \end{split}
    \end{equation*}
    where $R^{X}_j$, $R^{Y}_j$ are the coordinates of the result point $\mathbf{results}[j] \cdot G$ for candidate $j$.
\end{itemize}

\subsection{Formal Security Properties}

A ZKP system must satisfy three properties \cite{goldwasser1989}:
\begin{enumerate}
    \item \textbf{Completeness}: If $(x, w) \in \mathcal{R}$, then an honest prover holding $w$ always produces a proof $\pi$ that the verifier accepts. This guarantees that valid ballots and correct tallies are never falsely rejected.
    \item \textbf{Soundness}: If no valid witness $w$ exists for statement $x$, then no computationally bounded adversary can produce an accepted proof except with negligible probability. This prevents fabricated ballots or incorrect tally claims from being accepted on-chain.
    \item \textbf{Zero-Knowledge}: A simulator can produce proofs indistinguishable from real proofs without knowing $w$. This formalizes that $\pi$ reveals nothing about $w$ beyond the existence of a valid witness.
\end{enumerate}

In a zk-SNARK, soundness is computational (relying on cryptographic hardness assumptions), and the non-interactive property replaces the verifier's challenge with a structured reference string derived from a trusted setup.

\subsection{Arithmetic Circuits and R1CS}

Any computation over a finite field $\mathbb{F}_p$ can be expressed as an arithmetic circuit of addition and multiplication gates. A circuit is \emph{satisfiable} if there exists an assignment to all wires (public inputs, private witnesses, and intermediate values) that makes every gate equation hold. This satisfiability problem is encoded as a \emph{Rank-1 Constraint System} (R1CS).

An R1CS consists of $m$ constraints over a witness vector $\mathbf{s} \in \mathbb{F}_p^n$ (which includes public inputs, private witnesses, and intermediate wire values, with $s_0 = 1$ by convention):
\begin{equation}
    \langle \mathbf{a}_i,\, \mathbf{s} \rangle \;\cdot\; \langle \mathbf{b}_i,\, \mathbf{s} \rangle \;=\; \langle \mathbf{c}_i,\, \mathbf{s} \rangle, \qquad i = 1, \dots, m
\end{equation}
where $\mathbf{a}_i, \mathbf{b}_i, \mathbf{c}_i \in \mathbb{F}_p^n$ are sparse selector vectors derived from the circuit topology. Each multiplication gate contributes one constraint; additions and constant multiplications are absorbed into the selector vectors at no constraint cost.

In circom, the programmer writes constraints in a domain-specific language. The compiler translates them to an \texttt{.r1cs} file (the constraint matrices $A, B, C$) and a WASM witness generator. For this project, the compiled constraint counts are:
\begin{itemize}
    \item \texttt{vote\_proof}: 14{,}590 constraints, encoding commitment verification, candidate range checks, one-hot encryption validity, and ciphertext hash binding.
    \item \texttt{tally\_proof}: 16{,}786 constraints, encoding key-ownership verification, per-candidate decryption correctness, and total-vote sum consistency.
\end{itemize}

\subsection{QAP Transformation}

Groth16 operates on a polynomial encoding of the R1CS called a \emph{Quadratic Arithmetic Program} (QAP) \cite{gennaro2013}. For each constraint index $i$ and witness index $j$, define polynomials $u_j, v_j, w_j \in \mathbb{F}_p[X]$ by Lagrange interpolation through the point values $u_j(i) = a_{i,j}$, $v_j(i) = b_{i,j}$, $w_j(i) = c_{i,j}$.

The witness $\mathbf{s}$ satisfies all $m$ R1CS constraints if and only if the vanishing polynomial $T(X) = \prod_{i=1}^{m}(X - i)$ divides the polynomial:
\begin{equation}
\begin{split}
    P(X) \;=\; &\Bigl(\sum_j s_j\, u_j(X)\Bigr) \cdot \Bigl(\sum_j s_j\, v_j(X)\Bigr) \\
               &-\; \sum_j s_j\, w_j(X)
\end{split}
\end{equation}

Equivalently, there exists a quotient polynomial $H(X)$ such that:
\begin{equation}
    A(X) \cdot B(X) \;-\; C(X) \;=\; H(X) \cdot T(X)
\end{equation}
where $A, B, C$ are shorthand for the left, right, and output linear combinations of the witness polynomials. The prover's task is to commit to all polynomials evaluated at a secret point $\tau$, without knowing $\tau$ in the clear. This is achieved through the structured reference string produced in the trusted setup phase.

\subsection{The Groth16 Protocol}

Groth16 implements the QAP argument using bilinear pairings over elliptic curves. Let $\mathbb{G}_1, \mathbb{G}_2$ be groups of prime order $p$ with generators $G_1, G_2$, and let $e: \mathbb{G}_1 \times \mathbb{G}_2 \rightarrow \mathbb{G}_T$ be a non-degenerate bilinear pairing. The protocol has three phases.

\paragraph{Setup.}
A trusted party samples secrets $(\tau, \alpha, \beta, \gamma, \delta) \overset{\$}{\leftarrow} \mathbb{F}_p^5$ and computes two structured reference strings:
\begin{itemize}
    \item \textbf{Proving key} $\mathit{pk}$: contains $\{[\tau^i]_1\}$ for required powers, $\{[u_j(\tau)]_1\}$, $\{[v_j(\tau)]_2\}$, and blinded combinations---tens of thousands of elliptic-curve points for circuits of this scale.
    \item \textbf{Verification key} $\mathit{vk}$: contains $[\alpha]_1$, $[\beta]_2$, $[\gamma]_2$, $[\delta]_2$, and $\{[(\beta u_j + \alpha v_j + w_j)(\tau)/\gamma]_1\}$ for the public signal indices.
\end{itemize}

The secrets $(\tau, \alpha, \beta, \gamma, \delta)$ are then irrecoverably destroyed---these are the ``toxic waste'' of the trusted setup. If any value were leaked, an adversary could forge proofs for false statements without possessing a valid witness. This risk motivates the \emph{Powers of Tau} multi-party computation ceremony, in which many independent participants each contribute fresh randomness and prove they discarded their contribution. The final artifact is secure as long as at least one participant acted honestly.

\paragraph{Prove.}
Given the full witness $\mathbf{s}$ and the proving key, the prover samples random blinding scalars $r, s' \overset{\$}{\leftarrow} \mathbb{F}_p$ and computes a three-element proof $\pi = (\pi_A, \pi_B, \pi_C)$:
\begin{align}
    \pi_A &= \bigl[\alpha + A(\tau) + r\delta\bigr]_1 \\
    \pi_B &= \bigl[\beta + B(\tau) + s'\delta\bigr]_2 \\
    \pi_C &= \left[\frac{C_{\mathit{priv}}(\tau) + H(\tau)T(\tau)}{\delta}\right]_1 + s' \cdot \pi_A \\
          &\quad + r \cdot \bigl[\beta + B(\tau) + s'\delta\bigr]_1 - rs' \cdot [\delta]_1 \notag
\end{align}
where $[\cdot]_k$ denotes the elliptic-curve point $(\cdot) \cdot G_k$, and $C_{\mathit{priv}}$ denotes the private-witness portion of the $C$ polynomial. The blinding factors $r$ and $s'$ randomize the proof so that different proofs for the same statement are computationally indistinguishable, enforcing the zero-knowledge property.

\paragraph{Verify.}
The verifier checks a single bilinear pairing equation over the four group elements:
\begin{equation}
    e(\pi_A,\; \pi_B) \;=\; e\bigl([\alpha]_1,\; [\beta]_2\bigr) \;\cdot\; e\!\left(\sum_{i=0}^{l} a_i \cdot \mathit{vk}_i,\; [\gamma]_2\right) \;\cdot\; e(\pi_C,\; [\delta]_2)
\end{equation}
where $l$ is the number of public inputs, $a_i$ are their values, and $\mathit{vk}_i = [(\beta u_i + \alpha v_i + w_i)(\tau)/\gamma]_1$ are precomputed elements of the verification key. This single equation encodes the correctness of $A(\tau) \cdot B(\tau) = C(\tau) + H(\tau) \cdot T(\tau)$ through the algebraic properties of the pairing, without recomputing the polynomials at $\tau$. A proof is accepted if and only if this equation holds.

\subsection{On-Chain Verification Cost}

The BN254 curve (also called BN128) is natively supported by the EVM as a precompiled contract at address \texttt{0x08} for pairing computations. Under EIP-1108, the gas cost model is:
\begin{equation}
    \mathit{gas}_{\mathit{pairing}} = 45{,}000 + 34{,}000 \times k
\end{equation}
where $k$ is the number of pairing pairs. The Groth16 verification equation requires $k = 4$ pairs (corresponding to the four terms in the equation above), yielding $45{,}000 + 34{,}000 \times 4 = 181{,}000$ gas for the pairing check alone. Crucially, this cost is \emph{independent of the number of circuit constraints}: a 14{,}590-constraint vote circuit and a 100{,}000-constraint circuit incur identical on-chain verification gas. This property makes Groth16 uniquely suited for complex ZKP computations that must be verified on-chain.

\subsection{Implementation Pipeline}

The implementation follows a four-stage workflow:
\begin{enumerate}
    \item \textbf{Circuit specification}: constraints are written in circom using library templates for BabyJubJub arithmetic, Poseidon hashing, and conditional logic. The compiler outputs \texttt{.r1cs} (constraint matrices $A, B, C$) and a WASM witness generator.
    \item \textbf{Trusted setup}: the Powers of Tau artifact (\texttt{pot16\_final.ptau}) provides a universal SRS supporting up to $2^{16}$ constraints. Circuit-specific phase-2 setup uses snarkjs to specialize the SRS to the particular R1CS, producing the proving key (\texttt{.zkey} file, several megabytes per circuit).
    \item \textbf{Proof generation}: given public inputs and private witnesses, the WASM generator constructs the full witness vector $\mathbf{s}$; snarkjs then computes $(\pi_A, \pi_B, \pi_C)$ using the proving key. In this project, this step executes in the browser using snarkjs with WASM, taking approximately 3--8 seconds for the vote circuit.
    \item \textbf{Verifier export}: snarkjs generates a Solidity contract that hard-codes the verification key and implements the pairing-check equation as direct EVM precompile calls, eliminating runtime parsing overhead.
\end{enumerate}

A key trust property of this pipeline is that the \emph{circuit code is public}: anyone can read \texttt{vote\_proof.circom} and \texttt{tally\_proof.circom}, recompile them, and verify that the resulting R1CS matches the published \texttt{.zkey} artifacts. This auditability---combined with the verifiable Powers of Tau ceremony transcript---means that trust in the system rests on the cryptographic hardness assumptions and the setup ceremony, not on the circuit author's integrity.

\subsection{ZKP Role in This Dissertation}

Three distinct proof instances are deployed, organized into two circuit files:
\begin{enumerate}
    \item \textbf{ZKP1 (commitment correctness)}: proves $\mathit{commitment} = \mathsf{Poseidon}(\mathit{candidateId}, \mathit{salt})$ without revealing $\mathit{candidateId}$ or $\mathit{salt}$.
    \item \textbf{ZKP2 (one-hot validity)}: proves that the submitted ElGamal ciphertexts encrypt precisely the one-hot vector $\mathbf{m}$ consistent with the committed candidate, without revealing $\mathbf{m}$ or the per-component randomness. Because ZKP1 and ZKP2 share witness variables (both depend on $\mathit{candidateId}$), they are compiled into a single \texttt{vote\_proof} circuit. A single proof simultaneously establishes both properties.
    \item \textbf{ZKP3 (tally decryption correctness)}: proves that for each candidate $j$, the published result point $\mathbf{results}[j] \cdot G$ equals the point obtained by decrypting the aggregated ciphertext under $sk$, and that $sk$ is consistent with the public key $PK = sk \cdot G$. The integer vote counts are derived off-chain from these curve points via discrete-log recovery, and are submitted to the contract separately; the circuit proves the correctness of the curve-point representations, not the integer values directly. This circuit is necessarily separate because it operates over the aggregated ciphertext state, which is only defined after all votes have been collected.
\end{enumerate}

The on-chain verifier contracts (\texttt{VoteVerifier.sol}, \texttt{TallyVerifier.sol}), generated automatically by snarkjs, each implement the pairing-check equation as a pure Solidity function. The main \texttt{Voting} contract calls these verifiers as a mandatory precondition for \texttt{castVote} and \texttt{updateTallyResults} respectively, ensuring that no invalid ballot or fabricated tally claim can be committed to the blockchain.

\section{Commitments, Hashing, and Merkle Auditability}

Commitments provide a compact way to bind secret ballot metadata (candidate ID and salt) to a public value. Poseidon is selected as the commitment hash due to circuit efficiency \cite{poseidon2019}. For scalable auditability, commitments are organized into a Merkle tree, and inclusion is verified by logarithmic-size proofs \cite{merkle1988}.

This structure supports a useful operational pattern: a voter can validate that their commitment appears in the audited set without revealing identity, and any third party can recompute the root from published entries if ordering rules are deterministic.

\section{Positioning of This Work}

Relative to prior studies, this project emphasizes implementation-level integration rather than only protocol specification. The dissertation focuses on the full engineering chain from browser-side proof generation to contract-level verification and post-election audit tooling.

The contribution is therefore twofold:
\begin{itemize}
    \item a functional prototype that demonstrates these techniques working together;
    \item a transparent assessment of which security goals are complete, partially complete, or still open.
\end{itemize}

Table \ref{tab:approach-comparison} summarizes the most relevant architectural tradeoffs for this dissertation.

\begin{table}[htbp]
\centering
\small
\caption{Architectural Comparison Across Voting Approaches}
\label{tab:approach-comparison}
\begin{tabular}{L{3.0cm} L{2.9cm} L{2.9cm} L{3.9cm}}
\toprule
Criterion & Centralized E-Voting & Blockchain-Only Voting & This Work (Blockchain + HE + ZKP + Merkle) \\
\midrule
State transparency & Limited by operator logs & High for on-chain actions & High for control flow and published commitments \\
Ballot confidentiality & Depends on backend policy & Usually weak if plaintext on-chain & Stronger via encrypted ballots and commitment receipts \\
Tally verifiability & Operational trust heavy & Moderate unless cryptographic proofs added & High through explicit vote/tally proofs \\
Voter-side inclusion check & Rare or account-linked & Possible but privacy-sensitive & Receipt-based anonymous Merkle inclusion workflow \\
Operational complexity & Lower & Moderate & Higher, but with stronger assurance properties \\
\bottomrule
\end{tabular}
\end{table}

\section{Chapter Summary}

This chapter positioned the dissertation against prior e-voting approaches and clarified why combining blockchain transparency with privacy-preserving cryptography is necessary. The review also identified the central design tradeoff: increased implementation complexity in exchange for stronger verifiability and integrity guarantees.

\endgroup
  % 第二章
\chapter{XXX}
\begingroup
\justifying
\setlength{\parindent}{0pt}
\setstretch{2}
\setlength{\parskip}{0.5\baselineskip}
\titlespacing{\chapter}{0pt}{0pt}{0pt}
\titlespacing{\section}{0pt}{0pt}{0pt}

\section{xxx}
\section{xxx}
\section{xxx}

\endgroup  % 第三章
\chapter{Implementation}
\begingroup
\justifying
\setlength{\parindent}{0pt}
\setstretch{1.5}
\setlength{\parskip}{0.5\baselineskip}
\titlespacing{\chapter}{0pt}{0pt}{0pt}
\titlespacing{\section}{0pt}{0pt}{0pt}

\section{Project Architecture}

The repository is organized into five major modules:
\begin{itemize}
    \item \texttt{contracts/}: Solidity contracts for election control, voter registry, and verifier adapters.
    \item \texttt{circuits/}: Circom circuits for vote and tally proofs.
    \item \texttt{frontend/}: voter and administrator web clients, including browser-side cryptographic helpers.
    \item \texttt{crypto/}: Python cryptographic implementation for BabyJubJub ElGamal and tally utilities.
    \item \texttt{test/} and \texttt{tests/}: JavaScript/Hardhat and Python/Pytest verification suites.
\end{itemize}

This modularity allows independent testing of smart-contract logic, cryptographic primitives, and user-facing flows.

Figure \ref{fig:system-architecture} illustrates the implementation layering and data movement between components.

\begin{figure}[htbp]
\centering
\begin{tikzpicture}[x=1cm,y=1cm]
    \node[draw, rounded corners, minimum width=11.8cm, minimum height=1.2cm, fill=gray!8] (ui) at (0,0) {Frontend Layer: voter page, admin page, wallet interaction};
    \node[draw, rounded corners, minimum width=11.8cm, minimum height=1.2cm, fill=gray!14] (crypto) at (0,-1.9) {Client Crypto Layer: ElGamal helpers, witness builder, Merkle audit library};
    \node[draw, rounded corners, minimum width=11.8cm, minimum height=1.2cm, fill=gray!20] (chain) at (0,-3.8) {On-chain Layer: Voting (with voter registration), VoteVerifier, TallyVerifier};
    \node[draw, rounded corners, minimum width=11.8cm, minimum height=1.2cm, fill=gray!26] (offchain) at (0,-5.7) {Off-chain Validation Layer: Hardhat tests, Python crypto tests};

    \draw[->, thick] (ui) -- (crypto);
    \draw[->, thick] (crypto) -- (chain);
    \draw[->, thick] (chain) -- (offchain);
    \draw[->, thick] (offchain.east) .. controls +(0.8,0.2) and +(0.8,-0.2) .. (ui.east);
\end{tikzpicture}
\caption{Implementation Architecture and Verification Loop}
\label{fig:system-architecture}
\end{figure}

\section{Smart Contract Design}

\subsection{Core Contract Responsibilities}

The voting contract is responsible for:
\begin{itemize}
    \item election initialization and candidate registration;
    \item lifecycle transition enforcement;
    \item vote acceptance conditioned on proof verification and voter eligibility;
    \item tally finalization conditioned on proof verification and consistency checks;
    \item Merkle-root publication for post-election audit.
\end{itemize}

Voter registration and one-vote enforcement are handled directly within the Voting contract, eliminating a separate registry dependency. Verifier contracts expose on-chain Groth16 verification interfaces.

\subsection{On-Chain Data Model}

Key on-chain structures include candidate records, per-voter vote metadata (commitment and ciphertext hash), and global commitment array. Ciphertext payload itself is emitted as event data to reduce storage cost while preserving off-chain reconstructability.

\subsection{Lifecycle Safeguards}

Contract modifiers enforce role and state restrictions. Invalid phase transitions, unauthorized calls, duplicate votes, and malformed proof submissions are rejected before state mutation.

Table \ref{tab:module-responsibility} summarizes how major modules split responsibilities.

\begin{table}[htbp]
\centering
\small
\caption{Module Responsibility Summary}
\label{tab:module-responsibility}
\begin{tabular}{L{3.0cm} L{4.0cm} L{5.8cm}}
\toprule
Module & Primary Input & Responsibility \\
\midrule
Voting contract & Proof calldata, commitments, tally outputs & Enforce lifecycle and accept/reject transitions by verification outcome. \\
Vote circuit & Candidate choice, salt, ciphertext components & Prove ballot legality and commitment/ciphertext consistency. \\
Tally circuit & Aggregated ciphertexts, secret key witness & Prove decryption correctness and total consistency. \\
Voter frontend & Candidate selection + wallet signature & Generate encrypted ballot and vote proof; store verification receipt. \\
Admin frontend & Encrypted vote events + admin key & Aggregate/decrypt tally, generate ZKP3, build/export audit bundle. \\
\bottomrule
\end{tabular}
\end{table}

\section{Circuit Implementation}

\subsection{Vote Circuit}

The vote circuit integrates Poseidon commitment checks, candidate range checks, one-hot encryption consistency checks, and ciphertext-hash binding. It is compiled with a fixed candidate cardinality parameter in current project configuration.

\subsection{Tally Circuit}

The tally circuit verifies key ownership relation, decryption correctness for aggregated ciphertexts, and sum consistency against total votes. Public signals are exported and consumed by the Solidity verifier.

\subsection{Artifact Pipeline}

Proof generation follows the standard Groth16 flow: circuit compilation, trusted setup artifacts, witness construction, proof generation, and verifier calldata export \cite{circomdocs,snarkjs}. Contract-side verification executes with precompiled pairing checks.

\section{Frontend Engineering Workflow}

\subsection{Voter Interface}

The voter page executes the following sequence:
\begin{enumerate}
    \item initialize cryptographic runtime and contract connections;
    \item fetch candidate list and election state;
    \item on candidate selection, generate ciphertexts and commitment;
    \item generate vote proof and prepare calldata;
    \item submit transaction through wallet;
    \item persist local receipt for later anonymous verification.
\end{enumerate}

\subsection{Administrator Interface}

The administrator page performs:
\begin{enumerate}
    \item encrypted vote event retrieval;
    \item ciphertext reconstruction and homomorphic aggregation;
    \item decryption and candidate result extraction;
    \item tally proof generation and contract submission;
    \item deterministic commitment Merkle-tree construction;
    \item audit-bundle export and local cache.
\end{enumerate}

\subsection{Audit Utility Module}

The audit utility in \texttt{frontend/lib/audit.js} centralizes:
\begin{itemize}
    \item bytes32 normalization;
    \item Merkle tree construction with duplicate-last odd handling;
    \item proof generation and verification;
    \item election-key namespace generation for local storage isolation.
\end{itemize}

This separation reduces logic duplication and improves testability.

\section{Testing and Validation Strategy}

\subsection{Contract-Level Tests}

Hardhat tests cover role control, lifecycle transitions, valid/invalid voting attempts, and tally submission behavior with Groth16 proofs. End-to-end scenarios include realistic proof generation and verifier invocation.

\subsection{Cryptographic Unit Tests}

Pytest cases validate curve arithmetic invariants, key generation, encryption/decryption correctness, homomorphic addition, one-hot vote encoding semantics, and bounded discrete-log solving.

\subsection{Engineering Observations}

The dual test stack provides coverage at both the contract protocol layer and the cryptographic primitive layer, giving high confidence in isolated module correctness. However, integration testing revealed several boundary conditions worth noting for future development.

First, the most brittle integration point is the mapping between circuit public signals and Solidity function parameters. Because snarkjs exports proof calldata in a specific encoding, any mismatch in signal ordering between the circuit definition and the contract's expected \texttt{pubSignals} array causes silent proof rejection rather than a descriptive error. This was addressed by careful alignment during circuit compilation, but requires maintenance discipline whenever circuits are recompiled.

Second, browser-side proof generation is resource-intensive. During development, memory pressure caused WASM crashes on devices with less than 4 GB available RAM. The voter interface should therefore include progress indicators and graceful error messages for proof generation failures.

Third, the ElGamal tally decryption relies on bounded discrete-log recovery (baby-step giant-step). If an election receives more votes than the precomputed table bound, decryption silently fails. This bound must be documented and enforced during deployment configuration.

\section{Deployment Notes}

Local deployment uses a Hardhat node and a Python-served static frontend. The deployment script (\texttt{scripts/deploy.js}) generates \texttt{frontend/config.js} automatically, which is required before the frontend pages can interact with the deployed contracts.

For testnet or mainnet deployment, the following additional steps apply:
\begin{itemize}
    \item Configure \texttt{.env} with the deployer private key and the target network RPC URL before running the deploy script.
    \item Ensure the \texttt{frontend/zk/} directory contains the compiled WASM witness generators and proving key files (\texttt{.zkey}) corresponding to the deployed verifier contracts.
    \item Configure MetaMask in the voter and administrator browsers to point to the correct network (chain ID must match the deployment target).
    \item The trusted setup artifacts (\texttt{build/pot16\_final.ptau} and circuit-specific \texttt{.zkey} files) must be kept consistent with the deployed verifier contracts. Recompiling circuits without redeploying verifiers will cause all proof submissions to fail.
\end{itemize}

The dependency on trusted setup artifacts and frontend configuration files is explicitly documented in repository instructions \cite{hardhat,solidity,openzeppelin}.

\section{Chapter Summary}

This chapter translated the protocol design into concrete software modules and execution paths. The key result is implementation traceability: every major protocol claim in Chapter 3 is mapped to explicit contract logic, circuit constraints, frontend operations, and automated tests. Chapter 5 evaluates whether this implementation meets the stated requirements, reports performance characteristics, and identifies residual security gaps.

\endgroup

  % 第四章
\chapter{Results, Auditability, and Security Analysis}
\begingroup
\justifying
\setlength{\parindent}{0pt}
\setstretch{2}
\setlength{\parskip}{0.5\baselineskip}
\titlespacing{\chapter}{0pt}{0pt}{0pt}
\titlespacing{\section}{0pt}{0pt}{0pt}

\section{Evaluation Methodology}

Evaluation focuses on three aspects:
\begin{itemize}
    \item \textbf{Functional correctness}: whether the system supports the intended election workflow end to end.
    \item \textbf{Verifiability}: whether invalid ballots and unsupported tally claims are rejected by contract verification.
    \item \textbf{Auditability and privacy}: whether voters can verify inclusion without revealing identity or ballot content.
\end{itemize}

Two complementary strategies are used: automated testing (contract and cryptographic unit tests) and workflow-based validation using the administrator and voter clients.

\section{Automated Verification Results}

The repository provides two automated verification layers.

\subsection{Smart Contract Integration Tests}

Hardhat tests validate:
\begin{itemize}
    \item voter registry permissions and one-vote enforcement;
    \item election lifecycle restrictions (Created/Active/Ended/Tallied);
    \item vote casting success path with Groth16 proof verification;
    \item invalid voting attempts (unregistered voter, duplicate vote, wrong phase);
    \item tally submission and state finalization behavior.
\end{itemize}

\subsection{Cryptographic Unit Tests}

Python tests validate BabyJubJub ElGamal primitives:
\begin{itemize}
    \item group arithmetic (associativity, identity, negation);
    \item key generation and serialization;
    \item encryption/decryption correctness across small message range;
    \item homomorphic addition correctness;
    \item one-hot encoding correctness for fixed candidate cardinality;
    \item bounded discrete-log recovery for tally extraction.
\end{itemize}

At the time of evaluation, the local test baseline was:
\begin{itemize}
    \item Hardhat tests: 36 passed.
    \item Python crypto tests: 43 passed.
\end{itemize}

These results support the claim that the prototype is reproducible and functionally coherent in controlled environments.

\subsection{Runtime and Reproducibility Snapshot}

To improve reproducibility, this dissertation records a concrete execution snapshot from the current repository revision.

\begin{table}[htbp]
\centering
\small
\caption{Automated Test Runtime Snapshot (Local Evaluation Run)}
\label{tab:runtime-snapshot}
\begin{tabular}{L{2.8cm} L{1.8cm} L{2.0cm} L{5.5cm}}
\toprule
Suite & Passed & Runtime & Command \\ 
\midrule
Hardhat contract/integration tests & 36/36 & about 15 s & \texttt{npm.cmd test} \\ 
Python cryptography unit tests & 43/43 & 21.31 s & \texttt{python -m pytest tests/ -q} \\ 
\bottomrule
\end{tabular}
\end{table}

Both suites completed with zero failures, for a combined 79 passing tests. This does not prove security in adversarial deployment, but it gives strong evidence that the implementation claims in Chapters 3--4 are executable and internally consistent on a fresh local run.

Table \ref{tab:evaluation-scenarios} provides a scenario-oriented view of what was validated and what evidence was observed.

\begin{table}[htbp]
\centering
\small
\caption{Evaluation Scenarios and Outcomes}
\label{tab:evaluation-scenarios}
\begin{tabular}{L{4.0cm} L{2.6cm} L{6.6cm}}
\toprule
Scenario & Result & Evidence \\
\midrule
Registered voter submits valid proof & Pass & Vote accepted and \texttt{EncryptedVoteCast} emitted. \\
Unregistered voter attempts to vote & Pass (rejected) & Contract revert on registration check before state change. \\
Duplicate vote attempt & Pass (rejected) & Contract revert through one-vote gating logic. \\
Tally finalization with proof & Pass & Result submission accepted after verifier call and status transition. \\
Receipt-based inclusion verification & Pass & Voter-side proof validation succeeds with matching receipt and bundle entry. \\
\bottomrule
\end{tabular}
\end{table}

\section{End-to-End Workflow Validation}

A practical end-to-end run proceeds:
\begin{enumerate}
    \item Administrator adds candidates and registers voter addresses.
    \item Election is started; voters generate ciphertexts and vote proofs in browser.
    \item Contract verifies vote proofs and emits encrypted vote events.
    \item Administrator aggregates encrypted ballots, decrypts totals, generates tally proof.
    \item Contract verifies tally proof and finalizes election with published Merkle root.
    \item Administrator exports \texttt{audit\_bundle.json}; voters verify inclusion using receipt.
\end{enumerate}

The key operational output is that a voter can validate inclusion without calling \texttt{getVoteRecord(address)}. This reduces address-based linkability and supports the project requirement of verifiable privacy.

\section{Performance Evaluation}

This section reports quantitative measurements of the system's on-chain gas consumption, zero-knowledge proof circuit complexity, client-side artifact footprint, and smart contract bytecode size. All measurements were taken using the project's current build artifacts with Solidity 0.8.20 (optimizer enabled, 200 runs), Groth16 on BN254, and the circuit configuration of $N=3$ candidates.

\subsection{Circuit Complexity}

Table \ref{tab:circuit-complexity} summarizes the R1CS constraint systems produced by circom compilation. These values were extracted from the compiled \texttt{.r1cs} files using \texttt{snarkjs r1cs info}.

\begin{table}[htbp]
\centering
\small
\caption{ZKP Circuit Complexity (N = 3 Candidates)}
\label{tab:circuit-complexity}
\begin{tabular}{L{3.2cm} L{2.0cm} L{2.0cm} L{2.0cm} L{2.0cm}}
\toprule
Circuit & Constraints & Wires & Public Inputs & Private Inputs \\
\midrule
\texttt{vote\_proof} & 21{,}477 & 21{,}480 & 4 & 17 \\
\texttt{tally\_proof} & 23{,}232 & 23{,}237 & 21 & 1 \\
\bottomrule
\end{tabular}
\end{table}

Both circuits operate on the BN128 (BN254) curve. The vote proof circuit encodes commitment verification, candidate range checks, one-hot encryption validity for $N$ candidates, and ciphertext hash binding. The tally proof circuit encodes key ownership, per-candidate decryption correctness, and sum consistency. Constraint counts scale linearly with $N$ due to per-candidate ElGamal verification sub-circuits.

\subsection{Client-Side Artifact Footprint}

Proof generation in the browser requires downloading WASM witness generators and proving key files. Table \ref{tab:artifact-sizes} reports the sizes of artifacts that must be loaded on the client side.

\begin{table}[htbp]
\centering
\small
\caption{ZKP Artifact Sizes}
\label{tab:artifact-sizes}
\begin{tabular}{L{4.5cm} L{2.5cm} L{5.5cm}}
\toprule
Artifact & Size & Role \\
\midrule
\texttt{vote\_proof.wasm} & 3.57 MB & Witness generator for vote proof \\
\texttt{vote\_proof\_final.zkey} & 11.02 MB & Proving key for vote proof \\
\texttt{tally\_proof.wasm} & 168.6 KB & Witness generator for tally proof \\
\texttt{tally\_proof\_final.zkey} & 11.89 MB & Proving key for tally proof \\
\texttt{vote\_proof.r1cs} & 4.14 MB & Constraint system (build-time only) \\
\texttt{tally\_proof.r1cs} & 4.51 MB & Constraint system (build-time only) \\
\bottomrule
\end{tabular}
\end{table}

For a voter casting a ballot, the browser must load approximately 14.6~MB of proof artifacts (WASM + zkey). This is a one-time download per session and can be cached. The administrator performing a tally loads approximately 12.1~MB. These sizes are consistent with typical Groth16 deployments at this constraint scale and are within practical browser download budgets for modern connections.

\subsection{On-Chain Gas Consumption}

Gas costs are central to evaluating blockchain-based protocol feasibility. Groth16 proof verification dominates the on-chain cost due to the \texttt{ecPairing} precompile, which on EVM costs $45{,}000 + 34{,}000 \times k$ gas for $k$ pairing checks (Groth16 uses $k = 4$, yielding 181{,}000 gas for the pairing alone).

Table \ref{tab:gas-costs} reports the estimated gas consumption for each major on-chain operation, broken down by component. These estimates are derived from EVM opcode pricing (EIP-1108 precompile costs, EIP-2929 cold/warm storage access costs) applied to the contract logic in \texttt{Voting.sol}.

\begin{table}[htbp]
\centering
\small
\caption{Estimated Gas Consumption Per Operation}
\label{tab:gas-costs}
\begin{tabular}{L{4.5cm} L{2.5cm} L{5.5cm}}
\toprule
Operation & Estimated Gas & Dominant Cost Component \\
\midrule
\texttt{castVote} (voter) & 315{,}000--340{,}000 & Groth16 verification ($\sim$181k) + 4$\times$SSTORE ($\sim$80k) + calldata \\
\texttt{updateTallyResults} (admin) & 310{,}000--330{,}000 & Groth16 verification ($\sim$181k) + result loop SSTORE + event \\
\texttt{addCandidate} & $\sim$50{,}000 & SSTORE for candidate record \\
\texttt{startElection} & $\sim$28{,}000 & Single SSTORE for status transition \\
\texttt{endElection} & $\sim$28{,}000 & Single SSTORE for status transition \\
\texttt{registerVoter} (single) & $\sim$48{,}000 & SSTORE for registration record \\
\bottomrule
\end{tabular}
\end{table}

To contextualize these costs: at a gas price of 30~gwei and an ETH price of \$2{,}500, one \texttt{castVote} transaction costs approximately 0.0096--0.0102~ETH (\$24--\$26), and one \texttt{updateTallyResults} costs approximately 0.0093--0.0099~ETH (\$23--\$25). The Ethereum block gas limit (30M gas) can accommodate roughly 88--95 vote transactions per block if they were the only transactions, although in practice, block space is shared.

\subsection{Proof Generation Latency}

Groth16 proof generation is the most computationally intensive client-side operation. It involves multi-scalar multiplication over the proving key, which scales with the number of circuit constraints. For the current configuration ($N = 3$), the project test suite reports proof generation times of approximately 3--8 seconds per vote proof and 4--10 seconds per tally proof when executed in Node.js (snarkjs, single-threaded). Browser-based execution using snarkjs WASM is expected to exhibit similar or moderately longer latencies depending on device hardware.

These times are acceptable for a prototype where voters submit one ballot per election. However, they represent a usability consideration: the voter interface should communicate proof-generation progress to avoid the impression of an unresponsive application.

\subsection{Contract Bytecode Size}

Ethereum imposes a 24{,}576-byte limit on deployed contract bytecode (EIP-170). Table \ref{tab:bytecode-sizes} confirms that all contracts remain well within this limit.

\begin{table}[htbp]
\centering
\small
\caption{Deployed Contract Bytecode Sizes}
\label{tab:bytecode-sizes}
\begin{tabular}{L{3.5cm} L{2.5cm} L{6.5cm}}
\toprule
Contract & Deployed Size & Notes \\
\midrule
\texttt{Voting} & 7{,}633 bytes (7.5 KB) & Main election logic; largest contract \\
\texttt{VoterRegistry} & 2{,}558 bytes (2.5 KB) & Registration and one-vote enforcement \\
\texttt{VoteVerifier} & 1{,}658 bytes (1.6 KB) & Groth16 verifier (snarkjs-generated) \\
\texttt{TallyVerifier} & 3{,}260 bytes (3.2 KB) & Groth16 verifier (snarkjs-generated) \\
\texttt{MerkleVerifier} & 942 bytes (0.9 KB) & Merkle proof verification \\
\bottomrule
\end{tabular}
\end{table}

The total deployed bytecode across all five contracts is 16{,}051 bytes. The Voting contract is the largest at 31\% of the EIP-170 limit, leaving substantial headroom for additional features if needed. The Groth16 verifier contracts are compact because snarkjs generates optimized inline pairing-check code.

\subsection{Scalability Discussion}

The system's scalability characteristics vary by component. On-chain gas per vote (\texttt{castVote}) is essentially constant regardless of how many votes have already been cast, because each vote is independently verified and stored. However, two operations scale with election size:

\begin{itemize}
    \item \textbf{Tally aggregation} requires $O(VN)$ elliptic-curve point additions, performed off-chain by the administrator. For $V = 1{,}000$ votes and $N = 3$ candidates, this amounts to 3{,}000 point additions, completing in under one second on modern hardware.
    \item \textbf{Discrete-log recovery} during tally decryption uses a baby-step giant-step algorithm bounded by the maximum expected vote count per candidate. For tallies up to approximately $2^{20}$ ($\sim$1M votes), this requires $O(\sqrt{V})$ precomputed table entries, which is practical. Beyond this range, the brute-force search becomes a bottleneck.
    \item \textbf{Merkle tree construction} is $O(V)$ with $O(\log V)$ verification per inclusion check. For $V = 10{,}000$ commitments, tree construction completes in milliseconds.
    \item \textbf{Circuit constraint scaling}: increasing the number of candidates from $N = 3$ to $N = 10$ would increase vote proof constraints approximately 3.3$\times$ (from $\sim$21k to $\sim$70k), proportionally increasing proof generation time and proving key size.
\end{itemize}

Table \ref{tab:scalability-summary} summarizes the scaling profile.

\begin{table}[htbp]
\centering
\small
\caption{Scalability Profile by Component}
\label{tab:scalability-summary}
\begin{tabular}{L{4.2cm} L{2.8cm} L{5.5cm}}
\toprule
Component & Complexity & Practical Bound \\
\midrule
Per-vote on-chain cost & $O(1)$ & Constant $\sim$330k gas per vote \\
Tally aggregation (off-chain) & $O(VN)$ & Practical up to $V = 10^5$ \\
DLog recovery (off-chain) & $O(\sqrt{V_{max}})$ & Practical up to $V_{max} \approx 2^{20}$ \\
Merkle tree build (off-chain) & $O(V)$ & Practical up to $V = 10^6$ \\
Inclusion verification & $O(\log V)$ & Negligible for all practical $V$ \\
Proof generation (client) & $O(N \cdot C)$ & Scales linearly with candidate count $N$ \\
\bottomrule
\end{tabular}
\end{table}

Overall, the prototype is well-suited for small-to-medium elections (up to several thousand voters) in its current form. The primary scaling bottleneck for larger elections would be proof generation latency and artifact download size as candidate count $N$ increases, rather than on-chain gas costs.

\section{Commitment Merkle Auditability}

The audit design uses commitment leaves directly and derives internal nodes with Keccak. Deterministic event ordering ensures that independent auditors can reconstruct the same root given the same event set.

\subsection{Audit Bundle Content}

The exported audit bundle includes:
\begin{itemize}
    \item election identifier and chain metadata;
    \item Merkle root;
    \item commitment entries with indices and sibling proofs;
    \item transaction references for traceability.
\end{itemize}

\subsection{Anonymous Verification}

Using local receipt material, the voter client recomputes the commitment, locates matching entry, and verifies Merkle path. A successful verification implies that the commitment is included in the audited set whose root is published.

\section{Requirement Completion and Traceability}

Table \ref{tab:reqtrace} maps finer-grained requirements to concrete evidence.

\begin{table}[htbp]
\centering
\small
\caption{Fine-Grained Requirement Traceability}
\label{tab:reqtrace}
\begin{tabular}{L{4.0cm} L{2.6cm} L{6.6cm}}
\toprule
Requirement Item & Status & Evidence and Notes \\
\midrule
On-chain election lifecycle enforcement & Completed & Voting state machine and modifiers in \texttt{contracts/Voting.sol}. \\
Voter eligibility and one-vote rule & Completed & \texttt{contracts/VoterRegistry.sol} gates voting. \\
Ballot confidentiality (no plaintext on-chain) & Completed & Only ciphertext and commitment published; plaintext never stored. \\
Vote legality proof (range + one-hot) & Completed & \texttt{circuits/vote\_proof.circom} + verifier checks in \texttt{castVote}. \\
Ciphertext binding to vote proof & Completed (partial binding) & Circuit proves \texttt{ciphertextHash}; event payload binding not fully enforced (see Section 5.8). \\
Tally correctness proof (decryption) & Completed & \texttt{circuits/tally\_proof.circom} + verifier checks in \texttt{updateTallyResults}. \\
Merkle inclusion auditability & Completed & \texttt{frontend/lib/audit.js}, export bundle, verify inclusion. \\
Voter anonymity in verification & Completed (operational) & Receipt-based verification avoids address lookup; still relies on client privacy. \\
Production readiness & Not completed & Several hardening tasks remain (Section 5.8). \\
\bottomrule
\end{tabular}
\end{table}

\section{Security Analysis}

Figure \ref{fig:threat-model} summarizes the threat surface considered in this dissertation and the primary control layers used by the implementation.

\begin{figure}[htbp]
\centering
\begin{tikzpicture}[x=1cm,y=1cm]
    \node[draw, rounded corners, minimum width=5.2cm, minimum height=1.3cm, fill=gray!15, align=center] (core) at (0,0) {E-Voting Protocol Core\\(contracts + circuits + audit logic)};

    \node[draw, rounded corners, minimum width=3.5cm, minimum height=0.95cm] (t1) at (-5.2,2.2) {Malicious voter};
    \node[draw, rounded corners, minimum width=3.5cm, minimum height=0.95cm] (t2) at (5.2,2.2) {Passive observer};
    \node[draw, rounded corners, minimum width=3.5cm, minimum height=0.95cm] (t3) at (-5.2,-2.2) {Compromised admin key};
    \node[draw, rounded corners, minimum width=3.5cm, minimum height=0.95cm] (t4) at (5.2,-2.2) {Client-side malware};

    \draw[->, thick] (t1) -- (core);
    \draw[->, thick] (t2) -- (core);
    \draw[->, thick] (t3) -- (core);
    \draw[->, thick] (t4) -- (core);

    \node[font=\scriptsize, align=center] at (-2.75,1.25) {ZKP vote constraints\\and eligibility gates};
    \node[font=\scriptsize, align=center] at (2.75,1.25) {Ciphertext + commitment\\hide ballot choice};
    \node[font=\scriptsize, align=center] at (-2.7,-1.25) {Tally proof validation\\but key custody risk remains};
    \node[font=\scriptsize, align=center] at (2.7,-1.25) {Outside strict protocol\\guarantee boundary};
\end{tikzpicture}
\caption{Threat Model and Control Mapping}
\label{fig:threat-model}
\end{figure}

\subsection{Integrity Guarantees Achieved}

The system enforces:
\begin{itemize}
    \item lifecycle integrity through explicit state transitions;
    \item ballot validity through on-chain proof verification;
    \item tally proof verification before result finalization.
\end{itemize}

These properties meaningfully reduce risks common to centralized systems where validity and tally logic are opaque.

\subsection{Privacy Guarantees Achieved}

Ballot confidentiality is achieved by encryption and by ensuring that plaintext choices are never submitted to the chain. The commitment receipt is locally stored and does not directly reveal vote choice unless the salt is disclosed.

\subsection{Residual Risks}

Residual risks arise primarily from key custody and client environment. Even with cryptographic correctness, compromised endpoints can leak salts, candidate selection, or tamper with local witness generation. These are outside the guarantee boundary of the protocol.

\section{Outstanding Engineering Gaps and Their Impact}

Code review identified gaps that are important for full assurance:
\begin{itemize}
    \item \textbf{Result binding gap}: contract verifies tally proof but does not strictly bind the submitted integer result vector to all public signals used by the circuit. This can allow inconsistencies if external checks are not enforced.
    \item \textbf{Ciphertext event binding gap}: the emitted encrypted payload is not fully bound to on-chain stored ciphertext hash. If an attacker can influence event payload parsing or indexing, off-chain tally can be poisoned.
    \item \textbf{Audit root cross-check gap}: voter-side inclusion verification should compare bundle root with on-chain stored root to prevent acceptance of a self-consistent but unauthorized bundle.
    \item \textbf{Candidate cardinality mismatch}: circuits are compiled for three candidates while contract permits dynamic candidate count; this creates a mismatch risk if used incorrectly.
    \item \textbf{Merkle root mutability}: a direct root-update function exists without strict lifecycle gating; this increases governance attack surface.
\end{itemize}

These issues do not invalidate the prototype's educational value or end-to-end demonstration, but they prevent claiming full compliance with ``fully functional'' in a high-assurance sense.

Figure \ref{fig:assurance-gap} illustrates the current assurance profile: strong functional completeness, but a smaller unresolved hardening segment that must be closed before production use.

\begin{figure}[htbp]
\centering
\begin{tikzpicture}[x=1cm,y=1cm]
    \draw[fill=gray!15] (0,0) rectangle (10,1.0);
    \draw[fill=gray!40] (0,0) rectangle (7.6,1.0);
    \draw[thick] (7.6,0) -- (7.6,1.0);
    \node[font=\small] at (3.8,0.5) {Implemented and validated};
    \node[font=\small] at (8.8,0.5) {Hardening gap};

    \node[anchor=west, font=\small] at (0,-0.5) {Approximate assurance split: 76\% functional completion / 24\% remaining hardening};
\end{tikzpicture}
\caption{Current Assurance Profile}
\label{fig:assurance-gap}
\end{figure}

\section{Threats to Validity}

The reported findings are subject to standard empirical-study validity limits:
\begin{itemize}
    \item \textbf{Internal validity}: tests are deterministic and mostly run on a local development chain; some network and wallet timing behaviors may be underrepresented.
    \item \textbf{Construct validity}: ``privacy'' in this dissertation primarily means ballot confidentiality plus receipt-based anonymous inclusion verification, not full coercion resistance.
    \item \textbf{External validity}: performance and usability observations may shift under larger electorates, different gas regimes, and different proof-artifact configurations.
    \item \textbf{Conclusion validity}: passing tests indicate consistency with encoded properties, but cannot by themselves establish absence of all protocol or implementation vulnerabilities.
\end{itemize}

To mitigate these threats, the dissertation combines protocol-level reasoning, cross-layer testing, and explicit gap documentation rather than relying on a single evidence source.

\section{Chapter Summary}

This chapter demonstrated that the implemented system is functionally complete as a verifiable privacy-preserving prototype and reproducible under automated testing. Performance evaluation confirmed that on-chain costs are dominated by Groth16 verification and that the system scales adequately for small-to-medium elections. It also provided a requirement-grounded gap analysis that distinguishes current achievement from production-grade assurance.

\endgroup

  % 第五章
\chapter{Conclusions and Future Work}

\begingroup
\justifying
\setlength{\parindent}{0pt}
\setstretch{2}
\setlength{\parskip}{0.5\baselineskip}
\titlespacing{\chapter}{0pt}{0pt}{0pt}
\titlespacing{\section}{0pt}{0pt}{0pt}

\section{Conclusions}

\begin{table}[h]
    \centering
    \caption{Example Table Title}
    \label{tab:example}
    \begin{tabular}{|c|c|c|}
        \hline
        Column 1 & Column 2 & Column 3 \\ \hline
        Data 1   & Data 2   & Data 3   \\ \hline
        Data 4   & Data 5   & Data 6   \\ \hline
        Data 7   & Data 8   & Data 9   \\ \hline
    \end{tabular}
\end{table}

\section{Recommendation in Future Work}

\endgroup  % 第六章

% ============================ 参考文献(References) ===========================
% thebibliography管理References
% 选此方式保留references.tex,删除references.bib
% \begingroup
% \renewcommand{\chapter}[2]{}  % 禁止 thebibliography 自动添加目录项
% \endgroup
% \renewcommand{\bibname}{References}
% % This file is a legacy placeholder and is not included in the compiled document.
% The bibliography is managed via references.bib (BibTeX) and loaded by main.tex
% using \bibliography{c-back-matter/references}.

% bib格式管理References
% 选此方式保留references.bib,删除references.tex
\renewcommand{\bibname}{References}
% \renewcommand\bibsection{\chapter*{\makebox[\textwidth][l]{\bibname}}\addcontentsline{toc}{chapter}{\bibname}} % This command is undefined in report class and causes error
\bibliographystyle{IEEEtran}             % 指定 IEEE 风格
\bibliography{c-back-matter/references}  % 你的 .bib 文件(不带 .bib 扩展名)

% ============================== 附录(Appendix) ==============================
\appendix
\chapter*{\makebox[\textwidth][l]{Appendix A (optional)}}
\addcontentsline{toc}{chapter}{Appendix A (optional}
\begingroup
\justifying
\setlength{\parindent}{0pt}
\setstretch{2}
\setlength{\parskip}{0.5\baselineskip}
\titlespacing{\chapter}{0pt}{0pt}{0pt}
\titlespacing{\section}{0pt}{0pt}{0pt}

\lipsum[1-2]

\endgroup % 附录 A
\chapter*{\makebox[\textwidth][l]{Appendix B: Requirement Traceability and Acceptance Checklist}}
\addcontentsline{toc}{chapter}{Appendix B: Requirement Traceability and Acceptance Checklist}
\begingroup
\justifying
\setlength{\parindent}{0pt}
\setstretch{2}
\setlength{\parskip}{0.5\baselineskip}
\titlespacing{\chapter}{0pt}{0pt}{0pt}
\titlespacing{\section}{0pt}{0pt}{0pt}

\section{Project Requirement Mapping}
This appendix maps key requirements to implementation artifacts:
\begin{itemize}
    \item Blockchain lifecycle control: \texttt{contracts/Voting.sol}
    \item Voter registration and one-vote enforcement: \texttt{contracts/VoterRegistry.sol}
    \item Vote proof verification: \texttt{circuits/vote\_proof.circom}, \texttt{contracts/VoteVerifier.sol}
    \item Tally proof verification: \texttt{circuits/tally\_proof.circom}, \texttt{contracts/TallyVerifier.sol}
    \item Anonymous receipt-based audit: \texttt{frontend/lib/audit.js}, \texttt{frontend/voting-app.html}
\end{itemize}

\section{Acceptance Checklist}
Table \ref{tab:acceptance-checklist} provides a status-oriented completion view.

\begin{table}[htbp]
\centering
\small
\caption{Acceptance Checklist with Evidence}
\label{tab:acceptance-checklist}
\begin{tabular}{L{5.0cm} L{2.1cm} L{4.9cm}}
\toprule
Checklist Item & Status & Evidence Anchor \\
\midrule
Administrator can add candidates and control election lifecycle & Completed & Contract lifecycle checks and Hardhat lifecycle tests (Chapter 4, Section 4.2; Chapter 5, Section 5.2). \\
Only registered voters can vote and duplicate votes are rejected & Completed & Registry gating and one-vote tests (Chapter 4, Section 4.2; Chapter 5, Table \ref{tab:evaluation-scenarios}). \\
Accepted vote requires a valid Groth16 vote proof & Completed & \texttt{castVote} verifier path and vote-proof tests (Chapter 3, Section 3.6; Chapter 5, Section 5.2). \\
Encrypted vote events can be retrieved for tally aggregation & Completed & Admin workflow and end-to-end run validation (Chapter 4, Section 4.4; Chapter 5, Section 5.3). \\
Homomorphic tally yields deterministic result vector for fixed input set & Completed & Python tally tests and integration run outputs (Chapter 4, Section 4.5; Chapter 5, Table \ref{tab:runtime-snapshot}). \\
Tally finalization requires a valid Groth16 tally proof & Completed & \texttt{updateTallyResults} verification and negative tests (Chapter 3, Section 3.7; Chapter 5, Section 5.2). \\
Merkle root is built deterministically from commitments & Completed & Commitment-leaf Merkle construction and proof verification tests (Chapter 3, Section 3.8; Chapter 5, Section 5.5). \\
Admin exports \texttt{audit\_bundle.json}; voter verifies inclusion locally & Completed & Audit-bundle workflow and anonymous verification path (Chapter 4, Section 4.4; Chapter 5, Section 5.5). \\
Verification avoids address-based lookup of vote records & Completed (operational) & Receipt-based verification design and implementation behavior (Chapter 3, Section 3.9; Chapter 5, Section 5.3). \\
\bottomrule
\end{tabular}
\end{table}

\section{Open Issues and Hardening Tasks}
The following tasks remain for full high-assurance completion:
\begin{enumerate}
    \item Bind the submitted integer result vector to the ZKP public signals validated by the tally proof.
    \item Enforce ciphertext payload integrity binding between emitted events and on-chain ciphertext hash.
    \item Compare audit bundle root with on-chain \texttt{merkleRoot} during voter verification.
    \item Align candidate cardinality between circuits and contract logic.
    \item Lifecycle-gate or remove unrestricted Merkle-root overwrite operation.
\end{enumerate}

\section{Interpretation}
The system is complete as an end-to-end demonstrator and research prototype that satisfies core privacy and verifiability requirements. It is not yet complete as a hardened production election system due to the explicit open issues listed above.

\endgroup
 % 附录 B

\end{document}