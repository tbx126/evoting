\chapter*{\makebox[\textwidth][l]{Appendix B: Requirement Traceability and Acceptance Checklist}}
\addcontentsline{toc}{chapter}{Appendix B: Requirement Traceability and Acceptance Checklist}
\begingroup
\justifying
\setlength{\parindent}{0pt}
\setstretch{2}
\setlength{\parskip}{0.5\baselineskip}
\titlespacing{\chapter}{0pt}{0pt}{0pt}
\titlespacing{\section}{0pt}{0pt}{0pt}

\section{Project Requirement Mapping}
This appendix maps key requirements to implementation artifacts:
\begin{itemize}
    \item Blockchain lifecycle control: \texttt{contracts/Voting.sol}
    \item Voter registration and one-vote enforcement: \texttt{contracts/VoterRegistry.sol}
    \item Vote proof verification: \texttt{circuits/vote\_proof.circom}, \texttt{contracts/VoteVerifier.sol}
    \item Tally proof verification: \texttt{circuits/tally\_proof.circom}, \texttt{contracts/TallyVerifier.sol}
    \item Anonymous receipt-based audit: \texttt{frontend/lib/audit.js}, \texttt{frontend/voting-app.html}
\end{itemize}

\section{Acceptance Checklist}
Table \ref{tab:acceptance-checklist} provides a status-oriented completion view.

\begin{table}[htbp]
\centering
\small
\caption{Acceptance Checklist with Evidence}
\label{tab:acceptance-checklist}
\begin{tabular}{L{5.0cm} L{2.1cm} L{4.9cm}}
\toprule
Checklist Item & Status & Evidence Anchor \\
\midrule
Administrator can add candidates and control election lifecycle & Completed & Contract lifecycle checks and Hardhat lifecycle tests (Chapter 4, Section 4.2; Chapter 5, Section 5.2). \\
Only registered voters can vote and duplicate votes are rejected & Completed & Registry gating and one-vote tests (Chapter 4, Section 4.2; Chapter 5, Table \ref{tab:evaluation-scenarios}). \\
Accepted vote requires a valid Groth16 vote proof & Completed & \texttt{castVote} verifier path and vote-proof tests (Chapter 3, Section 3.6; Chapter 5, Section 5.2). \\
Encrypted vote events can be retrieved for tally aggregation & Completed & Admin workflow and end-to-end run validation (Chapter 4, Section 4.4; Chapter 5, Section 5.3). \\
Homomorphic tally yields deterministic result vector for fixed input set & Completed & Python tally tests and integration run outputs (Chapter 4, Section 4.5; Chapter 5, Table \ref{tab:runtime-snapshot}). \\
Tally finalization requires a valid Groth16 tally proof & Completed & \texttt{updateTallyResults} verification and negative tests (Chapter 3, Section 3.7; Chapter 5, Section 5.2). \\
Merkle root is built deterministically from commitments & Completed & Commitment-leaf Merkle construction and proof verification tests (Chapter 3, Section 3.8; Chapter 5, Section 5.5). \\
Admin exports \texttt{audit\_bundle.json}; voter verifies inclusion locally & Completed & Audit-bundle workflow and anonymous verification path (Chapter 4, Section 4.4; Chapter 5, Section 5.5). \\
Verification avoids address-based lookup of vote records & Completed (operational) & Receipt-based verification design and implementation behavior (Chapter 3, Section 3.9; Chapter 5, Section 5.3). \\
\bottomrule
\end{tabular}
\end{table}

\section{Open Issues and Hardening Tasks}
The following tasks remain for full high-assurance completion:
\begin{enumerate}
    \item Bind the submitted integer result vector to the ZKP public signals validated by the tally proof.
    \item Enforce ciphertext payload integrity binding between emitted events and on-chain ciphertext hash.
    \item Compare audit bundle root with on-chain \texttt{merkleRoot} during voter verification.
    \item Align candidate cardinality between circuits and contract logic.
    \item Lifecycle-gate or remove unrestricted Merkle-root overwrite operation.
\end{enumerate}

\section{Interpretation}
The system is complete as an end-to-end demonstrator and research prototype that satisfies core privacy and verifiability requirements. It is not yet complete as a hardened production election system due to the explicit open issues listed above.

\endgroup
