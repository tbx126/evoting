\chapter*{Abstract}
\addcontentsline{toc}{chapter}{Abstract}
\begingroup
\justifying
\setlength{\parindent}{0pt}
\setstretch{2}
\setlength{\parskip}{0.5\baselineskip}

This dissertation presents the design and implementation of a blockchain-based electronic voting system with verifiability and voter privacy. The system integrates three cryptographic mechanisms in a single end-to-end workflow: homomorphic ElGamal encryption over BabyJubJub for confidential ballots, Groth16 zero-knowledge proofs for vote and tally correctness, and commitment-based Merkle proofs for anonymous post-election auditability.

During vote casting, each voter encrypts a one-hot ballot vector, computes a commitment and ciphertext binding hash, and submits a proof that validates ballot legality and consistency without revealing the selected candidate. During tallying, encrypted ballots are homomorphically aggregated, decrypted totals are accompanied by a tally proof, and a Merkle root derived from on-chain commitments is published. Voters can then verify inclusion of their own ballot commitment using a local receipt and an exported audit bundle, without address-based lookup.

Implementation uses Solidity, Hardhat, Circom, snarkjs, Python cryptographic modules, and browser-side JavaScript clients. Evaluation via automated tests confirms functional end-to-end behavior and reproducibility. Requirement analysis shows that core functional goals are achieved, while several hardening tasks remain for production readiness, including stronger result binding, root consistency checks, and candidate-cardinality alignment between circuits and contract logic.

\endgroup
