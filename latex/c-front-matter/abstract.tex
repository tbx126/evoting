\chapter*{Abstract}
\addcontentsline{toc}{chapter}{Abstract}
\begingroup
\justifying
\setlength{\parindent}{0pt}
\setstretch{1.5}
\setlength{\parskip}{0.5\baselineskip}

This dissertation presents the design, implementation, and evaluation of a blockchain-based electronic voting system that enforces ballot confidentiality and election verifiability through cryptographic protocol rather than institutional trust. The system integrates three mechanisms in a single end-to-end pipeline: homomorphic ElGamal encryption on BabyJubJub for confidential one-hot ballots, Groth16 zero-knowledge proofs for on-chain vote legality and tally correctness verification, and commitment-based Merkle proofs for anonymous post-election auditability---a combination not previously assembled into a fully tested, open-source, browser-to-chain prototype.

During vote casting, each voter encrypts a one-hot ballot vector under the administrator's public key, computes a Poseidon commitment and ciphertext binding hash, and submits a Groth16 proof that validates ballot legality and ciphertext consistency without revealing the selected candidate. During tallying, encrypted ballots are homomorphically aggregated off-chain, the decrypted results are accompanied by a tally correctness proof, and a Merkle root derived from on-chain commitments is published. Voters can then verify inclusion of their own ballot commitment using a locally stored receipt and an exported audit bundle, without any address-based lookup.

The implementation uses Solidity and Hardhat for smart contracts, Circom and snarkjs for ZKP circuits, Python for cryptographic validation, and browser-side JavaScript for client-facing proving. Seventy-five automated tests---32 Hardhat contract tests and 43 Python cryptographic unit tests---all pass, showing the prototype works end to end without failures. Each \texttt{castVote} call costs roughly 315,000--340,000 gas on-chain; the bulk of that comes from the Groth16 pairing check at 181,000 gas, a number that stays flat regardless of circuit size. Both confidentiality and verifiability hold in practice, though five open tasks---result-vector binding, ciphertext payload integrity, and candidate cardinality alignment among them---still need closing before the system can be deployed.

\endgroup
