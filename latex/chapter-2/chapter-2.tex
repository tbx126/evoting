\chapter{Literature Review}
\begingroup
\justifying
\setlength{\parindent}{0pt}
\setstretch{2}
\setlength{\parskip}{0.5\baselineskip}
\titlespacing{\chapter}{0pt}{0pt}{0pt}
\titlespacing{\section}{0pt}{0pt}{0pt}

\section{Traditional E-Voting Architecture}
Conventional electronic voting systems usually rely on a centralized software stack for ballot submission, storage, and tallying. The practical advantage is deployment simplicity. The core limitation is trust concentration: integrity, availability, and auditability are coupled to one organization and one data pipeline. In this model, independent verification by voters and third parties is often constrained by platform-level access and log export policies.

\section{Blockchain and Verifiable Voting}
Blockchain-based voting systems move election-state transitions to smart contracts and expose immutable events for public inspection. This design improves transparency and tamper resistance of control flow. However, naive on-chain voting leaks ballot semantics. Therefore, confidentiality-preserving constructions are required to avoid replacing centralized trust with public privacy loss.

\section{Positioning of This Work}
This project combines three elements into one practical workflow: (i) homomorphic ElGamal encryption for confidential ballots, (ii) Groth16 proofs for vote and tally correctness checks, and (iii) commitment-based Merkle audit bundles for receipt-driven voter verification. The implementation goal is not only cryptographic privacy but also operational verifiability: a voter can verify inclusion using a local receipt and an exported audit bundle, without address-based lookup.

\endgroup
